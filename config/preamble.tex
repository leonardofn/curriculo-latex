% Define as regras tipográficas para o português do Brasil.
\usepackage[brazilian]{babel}

% Permite a codificação de caracteres UTF-8, essencial para acentos e caracteres especiais.
\usepackage[utf8]{inputenc}

% Configura a codificação de fontes para T1, suportando caracteres acentuados.
\usepackage[T1]{fontenc}

% Permite a inclusão de texto literal (como código) com os ambientes "verbatim" e "verbatim*".
\usepackage{verbatim}

% Fornece comandos para personalizar a formatação de seções (títulos).
\usepackage{titlesec}

% Permite o uso de cores no texto e fundo.
\usepackage{color}

% Oferece controle detalhado sobre a formatação de listas (enumerate, itemize, description).
\usepackage{enumitem}

% Permite a personalização de cabeçalhos e rodapés.
\usepackage{fancyhdr}

% Estende o ambiente "tabular" com colunas que se ajustam automaticamente à largura disponível.
\usepackage{tabularx}

% Fornece símbolos matemáticos adicionais.
\usepackage{latexsym}

% Inclui a fonte de símbolos de Martin Vogel, com diversos símbolos úteis.
\usepackage{marvosym}

% Define margens de 1 polegada e remove cabeçalhos e rodapés padrão.
\usepackage[empty]{fullpage}

% Remove a formatação padrão de hiperlinks (cores e caixas).
\usepackage[hidelinks]{hyperref}

% Fornece o comando "\ul" para sublinhar texto sem quebrar linhas.
\usepackage[normalem]{ulem}

% Permite centralizar o texto com maior flexibilidade.
\usepackage{ragged2e}

% Permite converter datas em números sequenciais para cálculos.
\usepackage{datenumber}
% ----------------------------------------------

% Converte nomes de glifos em Unicode para melhor compatibilidade com PDFs.
\input glyphtounicode

% Garante que os PDFs gerados sejam legíveis em sistemas de busca de texto.
\pdfgentounicode=1

% ----------------- OPÇÕES DE FONTE -----------------
% Usa a fonte Source Sans Pro como padrão.
\usepackage[default]{sourcesanspro}
% Garante que URLs usem a mesma fonte do restante do documento.
\urlstyle{same}
% --------------------------------------------------

% ----------------- OPÇÕES DE MARGEM -----------------
% Define o estilo da página para o configurado pelo pacote fancyhdr.
\pagestyle{fancy}
% Limpa todos os campos de cabeçalho e rodapé.
\fancyhf{}

% Remove a linha abaixo do cabeçalho.
\renewcommand{\headrulewidth}{0in}
% Remove a linha acima do rodapé.
\renewcommand{\footrulewidth}{0in}

% Remove o espaçamento entre colunas em tabelas.
\setlength{\tabcolsep}{0in}

% Ajusta as margens para 0.5 polegadas nas laterais e no topo.
\addtolength{\oddsidemargin}{-0.5in}
\addtolength{\topmargin}{-0.5in}

% Aumenta a largura e altura da área de texto em 1 polegada.
\addtolength{\textwidth}{1.0in}
\addtolength{\textheight}{1.0in}

% Evita espaçamento vertical extra no final das páginas.
\raggedbottom{}

% Evita espaçamento horizontal extra no final das linhas.
\raggedright{}
% ---------------------------------------------------

% ----------------- COMANDOS DE SEÇÃO -----------------
% \titleformat{<comando>}
%   [<forma>]{<formato>}{<rotulo>}{<separacao>}
%   {<codigo-antes>}[<codigo-depois>]

% comando: comando de seção a ser redefinido (ex: \section).
% forma: estilo da fonte (ex: scshape para letras maiúsculas pequenas).
% formato: formatação aplicada ao título inteiro (rótulo e texto).
% rotulo: define o rótulo da seção.
% separação: espaçamento horizontal entre o rótulo e o corpo do título.
% codigo-antes: codigo a ser executado antes do título.
% codigo-depois: codigo a ser executado após o título.

\titleformat{\section}{\scshape\large}{}{0em}{\color{blue}}[\color{black}\titlerule\vspace{0pt}]
% ----------------------------------------------------

% ----------------- REDEFINIÇÕES -----------------
% Redefine o estilo do marcador de lista (bullet point).
\renewcommand\labelitemii{$\vcenter{\hbox{\tiny$\bullet$}}$}

% Redefine a profundidade do sublinhado para 2pt.
\renewcommand{\ULdepth}{2pt}
% -------------------------------------------------

% ----------------- COMANDOS PERSONALIZADOS -----------------
% \vspace{} define um espaço vertical com o tamanho especificado.

% resumeItem: renderiza um item de lista (bullet point) para o currículo.
\newcommand{\resumeItem}[1]{%
  \item\small{#1}
}

% resumeItemListStart/End: inicia e finaliza a lista de itens do currículo.
\newcommand{\resumeItemListStart}{\begin{itemize}[rightmargin=0.11in]}
    \newcommand{\resumeItemListEnd}{\end{itemize}}

% resumeSectionType: renderiza um tipo de seção em negrito (ex: habilidades).
\newcommand{\resumeSectionType}[3]{%
  \item
  \begin{tabular*}{0.96\textwidth}[t]{%
    p{0.165\linewidth}p{0.02\linewidth}p{0.82\linewidth}
    }
    \textbf{#1} & #2 & #3
  \end{tabular*}\vspace{-2pt}
}

% resumeTrioHeading: renderiza três elementos em três colunas (ex: projetos).
\newcommand{\resumeTrioHeading}[3]{%
  \item
  \small{%
    \begin{tabular*}{0.96\textwidth}[t]{%
      l@{\extracolsep{\fill}}c@{\extracolsep{\fill}}r
      }
      \textbf{#1} & \textit{#2} & #3
    \end{tabular*}
  }
}

% resumeQuadHeading: renderiza quatro elementos em duas colunas (ex: experiência).
\newcommand{\resumeQuadHeading}[4]{%
  \item
  \begin{tabular*}{0.96\textwidth}[t]{l@{\extracolsep{\fill}}r}
    \textbf{#1} & #2 \\
    \textit{\small#3} & \textit{\small #4} \\
  \end{tabular*}
}

% resumeQuadHeadingChild: renderiza a segunda linha de resumeQuadHeading.
\newcommand{\resumeQuadHeadingChild}[2]{%
  \item
  \begin{tabular*}{0.96\textwidth}[t]{l@{\extracolsep{\fill}}r}
    \textbf{\small#1} & {\small#2} \\
  \end{tabular*}
}

% resumeHeadingListStart/End: inicia e finaliza a lista de cabeçalhos do currículo.
\newcommand{\resumeHeadingListStart}{%
  \begin{itemize}[leftmargin=0.15in, label={}]
    }
    \newcommand{\resumeHeadingListEnd}{\end{itemize}}

\newcounter{dateinitial}
\newcounter{datefinal}

% difftoday: calcula a diferença em anos entre uma data e a data atual.
\newcommand{\difftoday}[3]{%
  \setmydatenumber{dateinitial}{\the\year}{\the\month}{\the\day}
  \setmydatenumber{datefinal}{#1}{#2}{#3}%
  \addtocounter{datefinal}{-\thedateinitial}%
  \the\numexpr-\thedatefinal/365\relax\space % anos
  %\the\numexpr(-\thedatetwo - (-\thedatetwo/365)*365)/30\relax\space $ meses
}
