\documentclass[a4paper,12pt]{article}

% ------------------- PACOTES -------------------

% Define as regras tipográficas para o português do Brasil.
\usepackage[brazilian]{babel}

% Permite a codificação de caracteres UTF-8, essencial para acentos e caracteres especiais.
\usepackage[utf8]{inputenc}

% Configura a codificação de fontes para T1, suportando caracteres acentuados.
\usepackage[T1]{fontenc} 

% Permite a inclusão de texto literal (como código) com os ambientes "verbatim" e "verbatim*".
\usepackage{verbatim}

% Fornece comandos para personalizar a formatação de seções (títulos).
\usepackage{titlesec}

% Permite o uso de cores no texto e fundo.
\usepackage{color}

% Oferece controle detalhado sobre a formatação de listas (enumerate, itemize, description).
\usepackage{enumitem}

% Permite a personalização de cabeçalhos e rodapés.
\usepackage{fancyhdr}

% Estende o ambiente "tabular" com colunas que se ajustam automaticamente à largura disponível.
\usepackage{tabularx}

% Fornece símbolos matemáticos adicionais.
\usepackage{latexsym}

% Inclui a fonte de símbolos de Martin Vogel, com diversos símbolos úteis.
\usepackage{marvosym}

% Define margens de 1 polegada e remove cabeçalhos e rodapés padrão.
\usepackage[empty]{fullpage}

% Remove a formatação padrão de hiperlinks (cores e caixas).
\usepackage[hidelinks]{hyperref}

% Fornece o comando "\ul" para sublinhar texto sem quebrar linhas.
\usepackage[normalem]{ulem}

% Permite centralizar o texto com maior flexibilidade.
\usepackage{ragged2e}

% Permite converter datas em números sequenciais para cálculos.
\usepackage{datenumber}
% ----------------------------------------------

% Converte nomes de glifos em Unicode para melhor compatibilidade com PDFs.
\input glyphtounicode

% Garante que os PDFs gerados sejam legíveis em sistemas de busca de texto.
\pdfgentounicode=1

% ----------------- OPÇÕES DE FONTE -----------------
% Usa a fonte Source Sans Pro como padrão.
\usepackage[default]{sourcesanspro}
% Garante que URLs usem a mesma fonte do restante do documento.
\urlstyle{same}
% --------------------------------------------------

% ----------------- OPÇÕES DE MARGEM -----------------
% Define o estilo da página para o configurado pelo pacote fancyhdr.
\pagestyle{fancy}
% Limpa todos os campos de cabeçalho e rodapé.
\fancyhf{}

% Remove a linha abaixo do cabeçalho.
\renewcommand{\headrulewidth}{0in}
% Remove a linha acima do rodapé.
\renewcommand{\footrulewidth}{0in}

% Remove o espaçamento entre colunas em tabelas.
\setlength{\tabcolsep}{0in}

% Ajusta as margens para 0.5 polegadas nas laterais e no topo.
\addtolength{\oddsidemargin}{-0.5in}
\addtolength{\topmargin}{-0.5in}

% Aumenta a largura e altura da área de texto em 1 polegada.
\addtolength{\textwidth}{1.0in}
\addtolength{\textheight}{1.0in}

% Evita espaçamento vertical extra no final das páginas.
\raggedbottom{} 

% Evita espaçamento horizontal extra no final das linhas.
\raggedright{}
% ---------------------------------------------------

% ----------------- COMANDOS DE SEÇÃO -----------------
% \titleformat{<comando>}
%   [<forma>]{<formato>}{<rotulo>}{<separacao>}
%   {<codigo-antes>}[<codigo-depois>]

% comando: comando de seção a ser redefinido (ex: \section).
% forma: estilo da fonte (ex: scshape para letras maiúsculas pequenas).
% formato: formatação aplicada ao título inteiro (rótulo e texto).
% rotulo: define o rótulo da seção.
% separação: espaçamento horizontal entre o rótulo e o corpo do título.
% codigo-antes: codigo a ser executado antes do título.
% codigo-depois: codigo a ser executado após o título.

\titleformat{\section}{\scshape\large}{}{0em}{\color{blue}}[\color{black}\titlerule\vspace{0pt}]
% ----------------------------------------------------

% ----------------- REDEFINIÇÕES -----------------
% Redefine o estilo do marcador de lista (bullet point).
\renewcommand\labelitemii{$\vcenter{\hbox{\tiny$\bullet$}}$}

% Redefine a profundidade do sublinhado para 2pt.
\renewcommand{\ULdepth}{2pt}
% -------------------------------------------------

% ----------------- COMANDOS PERSONALIZADOS -----------------
% \vspace{} define um espaço vertical com o tamanho especificado.

% resumeItem: renderiza um item de lista (bullet point) para o currículo.
\newcommand{\resumeItem}[1]{%
	\item\small{#1}
}

% resumeItemListStart/End: inicia e finaliza a lista de itens do currículo.
\newcommand{\resumeItemListStart}{\begin{itemize}[rightmargin=0.11in]}
\newcommand{\resumeItemListEnd}{\end{itemize}}

% resumeSectionType: renderiza um tipo de seção em negrito (ex: habilidades).
\newcommand{\resumeSectionType}[3]{%
	\item
	\begin{tabular*}{0.96\textwidth}[t]{%
		p{0.165\linewidth}p{0.02\linewidth}p{0.82\linewidth}
	}
	\textbf{#1} & #2 & #3
	\end{tabular*}\vspace{-2pt}
}

% resumeTrioHeading: renderiza três elementos em três colunas (ex: projetos).
\newcommand{\resumeTrioHeading}[3]{%
	\item
	\small{%
		\begin{tabular*}{0.96\textwidth}[t]{%
				l@{\extracolsep{\fill}}c@{\extracolsep{\fill}}r
			}
			\textbf{#1} & \textit{#2} & #3
		\end{tabular*}
	}
}

% resumeQuadHeading: renderiza quatro elementos em duas colunas (ex: experiência).
\newcommand{\resumeQuadHeading}[4]{%
	\item
	\begin{tabular*}{0.96\textwidth}[t]{l@{\extracolsep{\fill}}r}
		\textbf{#1} & #2 \\
		\textit{\small#3} & \textit{\small #4} \\
	\end{tabular*}
}

% resumeQuadHeadingChild: renderiza a segunda linha de resumeQuadHeading.
\newcommand{\resumeQuadHeadingChild}[2]{%
	\item
	\begin{tabular*}{0.96\textwidth}[t]{l@{\extracolsep{\fill}}r}
		\textbf{\small#1} & {\small#2} \\
	\end{tabular*}
}

% resumeHeadingListStart/End: inicia e finaliza a lista de cabeçalhos do currículo.
\newcommand{\resumeHeadingListStart}{%
	\begin{itemize}[leftmargin=0.15in, label={}]
}
\newcommand{\resumeHeadingListEnd}{\end{itemize}}

\newcounter{dateinitial}
\newcounter{datefinal}

% difftoday: calcula a diferença em anos entre uma data e a data atual.
\newcommand{\difftoday}[3]{%
	\setmydatenumber{dateinitial}{\the\year}{\the\month}{\the\day}
	\setmydatenumber{datefinal}{#1}{#2}{#3}%
	\addtocounter{datefinal}{-\thedateinitial}%
	\the\numexpr-\thedatefinal/365\relax\space % anos
	%\the\numexpr(-\thedatetwo - (-\thedatetwo/365)*365)/30\relax\space $ meses
}
% ---------------------------------------------------


%-------------------------------------------------------
% VOCÊ PODE REORGANIZAR AS SEÇÕES NA ORDEM QUE PREFERIR
%-------------------------------------------------------


\begin{document}

%------------------CONTATOS------------------

% Verifique se todos os detalhes estão corretos. Você pode adicionar mais links, se necessário.

% Versão original
\begin{comment}

\begin{tabular*}{\textwidth}{l@{\extracolsep{\fill}}r}

  \textbf{\LARGE Leonardo Feitosa Nogueira\vspace{5pt}} & % row = 1, col = 1

  Parambu, CE, Brasil \\ % row = 1, col = 2

  \href{https://linkedin.com/in/jane-doe}{\uline{LinkedIn}} $|$ % row = 2, col = 1

  \href{https://github.com/jane-doe}{\uline{GitHub}} & % row = 2, col = 1

  \small{Email: \href{mailto:leonardofeitosa3010@gmail.com}{\uline{leonardofeitosa3010@gmail.com}}} $|$ % row = 2, col = 2

  Celular: (88) 99811-0397 \\ % row = 2, col = 2

\end{tabular*}

\end{comment}

% Versão adaptada
\begin{tabularx}{\textwidth}{>{\centering\arraybackslash}X} % Centraliza o texto na coluna X

  \huge{\textbf{Leonardo Feitosa Nogueira\vspace{5pt}}}                                                                      \\

  Parambu, CE, Brasil\vspace{2pt}                                                                                            \\

  \small{\href{mailto:leonardofeitosa3010@gmail.com}{\uline{leonardofeitosa3010@gmail.com}}} $|$ (88) 99811-0397\vspace{2pt} \\

  \href{https://www.linkedin.com/in/leo-feitosa}{\uline{LinkedIn}} $|$ \href{https://github.com/leonardofn}{\uline{GitHub}}  \\
\end{tabularx}

%--------------------------------------------


%-------------------PERFIL------------------

% Seja breve, simples e direto ao ponto.


\section{Perfil Profissional}

\small{
  Desenvolvedor de software com mais de \textbf{\difftoday{2021}{06}{01} anos de experiência} em \textbf{HTML, CSS, JavaScript e TypeScript}, buscando especialização em desenvolvimento full stack com foco em otimização de performance. Possuo conhecimentos sólidos em frameworks como \textbf{Angular} e \textbf{Ionic}, experiência com \textbf{APIs RESTful} e um forte interesse em me manter atualizado com as novas tecnologias do mercado. Objetivo aplicar minhas habilidades e experiência para contribuir em \textbf{projetos desafiadores e inovadores}, buscando aprimorar continuamente minhas competências e alcançar excelência técnica.
}

%--------------------------------------------


%------------------HABILIDADES------------------

% Adicione ou remova resumeSectionTypes de acordo com suas necessidades.


\section{Habilidades}

\resumeHeadingListStart{}

\resumeSectionType{Linguagens}{:}{HTML, CSS, JavaScript, TypeScript, Java}

\resumeSectionType{Frameworks}{:}{Angular, Ionic, Node.js, Express.js, Quarkus}

\resumeSectionType{Bibliotecas}{:}{RxJS, Material}

\resumeSectionType{Banco de Dados}{:}{PostgreSQL}

\resumeSectionType{Ferramentas}{:}{Visual Studio Code, IntelliJ IDEA, Postman, Maven, DBeaver, Docker, Git, GitHub, Azure DevOps}

\resumeHeadingListEnd{}

%--------------------------------------------

%------------------EXPERIÊNCIAS------------------

%  Destile todos os seus pontos de discussão em pequenos tópicos que sigam o padrão “desafio-ação-resultado” para obter o máximo de eficiência. Tente quantificar (usar números) seus pontos sempre que possível e destaque palavras importantes.


\section{Experiência}

\resumeHeadingListStart{}

\resumeQuadHeading{Desenvolvedor Full Stack Junior}{Set/2022 -- Atual}{Vetta Tecnologia}{Remoto -- Belo Horizonte, MG, Brasil}

\resumeItemListStart{}

\resumeItem{Atuei na migração de sistemas siderúrgicos obsoletos para tecnologias web modernas, resultando em uma melhoria na eficiência dos processos internos.}

\resumeItem{Preparei a estrutura inicial de novos sistemas, tanto frontend quanto backend, utilizando, respectivamente, arquitetura MVC e modular.}

\resumeItem{Implementei novas funcionalidades em sistemas existentes, aumentando a interação do usuário.}

\resumeItem{Corrigi bugs e resolvi problemas críticos, garantindo a operação contínua das aplicações.}

\resumeItem{Realizei integração entre sistemas por meio do consumo de APIs RESTful, facilitando o fluxo de dados entre as plataformas.}

\resumeItem{Dominei e apliquei Git Flow para a organização do versionamento de código, melhorando a colaboração da equipe.}

\resumeItem{Auxiliei, de forma técnica, outros membros da equipe em dúvidas sobre lógica e codificação, resolução de erros, aplicação de boas práticas de desenvolvimento, etc.}

\resumeItem{Utilizei metodologias ágeis nos processos de entrega das demandas, garantindo a entrega de projetos dentro do prazo aceitável.}

\resumeItemListEnd{}


\resumeQuadHeading{Desenvolvedor Web Frontend}{Jun/2021 -- Set/2022}{Elevar Commerce}{Remoto -- Fortaleza, CE, Brasil}

\resumeItemListStart{}

\resumeItem{Desenvolvi novas funcionalidades em websites e aplicativos móveis de comércio eletrônico dinâmicos e responsivos, utilizando HTML, CSS, JavaScript e TypeScript, com foco nos frameworks Angular e Ionic.}

\resumeItem{Trabalhei com APIs REST para recuperar e exibir dados dos bancos de dados.}

\resumeItem{Depurei e corrigi erros encontrados nas aplicações.}

\resumeItem{Contribui no desenvolvimento de novas funcionalidades e melhorias.}

\resumeItemListEnd{}


\resumeQuadHeading{Técnico em Informática}{Fev/2020 -- Fev/2021}{Secretaria de Desenvolvimento Científico e Tecnológico de Tauá}{Tauá, CE, Brasil}

\resumeItemListStart{}

\resumeItem{Configurei computadores, impressoras e roteadores, garantindo o funcionamento adequado da infraestrutura.}

\resumeItem{Prestei serviços de suporte no sistema informático e de internet.}

\resumeItemListEnd{}


\resumeQuadHeading{Estagiário}{Fev/2019 -- Mai/2019}{Duxx Soluções em Sistemas}{Remoto -- Fortaleza, CE, Brasil}

\resumeItemListStart{}

\resumeItem{Implementei funcionalidades em aplicativos móveis Ionic, utilizando HTML, CSS, JavaScript e TypeScript.}

\resumeItem{Utilizei Git e GitHub no processo de desenvolvimento, colaborando em projetos de equipe.}

\resumeItemListEnd{}


\resumeQuadHeading{Estagiário}{Fev/2018 -- Jul/2018}{NettVirtual Telecom}{Tauá, CE, Brasil}

\resumeItemListStart{}

\resumeItem{Instalei kits de acesso à internet via radiofrequência (roteador e antena).}

\resumeItem{Prestei serviço de suporte técnico.}

\resumeItem{Participei de treinamento sobre o funcionamento de infraestrutura de rede de provedores de internet.}

\resumeItemListEnd{}

\resumeHeadingListEnd{}

%---------------------------------------------

%------------------EDUCAÇÃO------------------

% Mencione sua formação acadêmica


\section{Educação}

\resumeHeadingListStart{}

\resumeQuadHeading{Instituto Federal de Educação, Ciência e Tecnologia do Ceará}{Mar/2016 -- Jun/2019}{Tecnólogo em Telemática}{Tauá, CE, Brasil}

\resumeHeadingListEnd{}

%---------------------------------------------


%------------------PROJETOS------------------

% Use resumeQuadHeading se quatro elementos forem viáveis (ex.: link de vídeo de demonstração); caso contrário, use resumeTrioHeading. Mantenha os tópicos simples e concisos e tente abranger uma ampla variedade de habilidades que você usou para desenvolver esses projetos.

\begin{comment}

\section{Projects}

\resumeHeadingListStart{}

\resumeTrioHeading{\href{https://project1.com}{\uline{Project 1}}}{React.js, Redux, PHP, MySQL Git}{\href{https://proect1.com/source-code/}{\uline{Source Code}}}

\resumeItemListStart{}

\resumeItem{Designed and developed a clean and modern website using \textbf{HTML, CSS, and JavaScript}}

\resumeItem{Optimized website for \textbf{speed and user experience}}

\resumeItem{Utilized \textbf{responsive design} to ensure compatibility across all devices}

\resumeItem{Deployed on GitHub pages via GitHub Actions}

\resumeItemListEnd{}


\resumeTrioHeading{Project 2}{Node.js, Express, JavaScript, Git}{\href{https:project2.com/source-code}{\uline{Source Code}}}

\resumeItemListStart{}

\resumeItem{A \textbf{CRUD application} exposed using a RESTful API made with Node.js}

\resumeItem{Exposed POST, GET, PATCH and DELETE HTTP methods using \textbf{Express}}

\resumeItemListEnd{}

\resumeHeadingListEnd{}

\end{comment}
%--------------------------------------------


%------------------OUTROS------------------

% Você pode adicionar suas realizações, elogios, certificações, etc.


\section{Certificações}

\resumeItemListStart{}

\resumeItem{API Restful Javascript com Node.js, Typescript, TypeORM, etc.}

\resumeItem{Laravel: Construindo APIs REST}

\resumeItem{Git e Github para Iniciantes}

\resumeItem{FullStack: API Runy on Rails + App Angular}

\resumeItem{Operação de Microcomputador}

\resumeItemListEnd{}

%--------------------------------------------

\end{document}
